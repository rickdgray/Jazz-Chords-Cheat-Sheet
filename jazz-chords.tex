\documentclass{article}

% default margins are gargantuan
\usepackage[bottom=1cm, right=1cm, left=1cm, top=1cm]{geometry}
\pagenumbering{gobble}

\usepackage{savesym}
\usepackage{multirow}
\usepackage{makecell}
\usepackage{leadsheets}
% leadsheets and gchords both have \chord so we have to save one to prevent collision
\savesymbol{chord}
\usepackage{gchords}
\usepackage{musixtex}

\def\musicintext#1#2{% normal
  {\let\extractline\relax
   \smallmusicsize \nobarnumbers
   \staffbotmarg0pt \setclef1{#1}
   \startextract\addspace{-\afterruleskip}#2\endextract}}

\begin{document}

\begin{tabular}{ | c | l | l | c | c | c | }
    \multicolumn{6}{c}{Jazz Chords} \\
    \hline
    Symbol & Chord & Theory & Notes & Piano & Guitar \\
    \hline
        \multirow{6}{*}{\writechord{Cmaj7}} &
        \multirow{6}{*}{Major Seventh} & &
        \multirowcell{6}{ notes image } &
        \multirowcell{6}{ piano image } &
        \multirowcell{6}{\chord{t}{n,f3p3,f2p2,n,f1p1,n}{C}} \\
        & & Root & & & \\
        & & Major Third & & & \\
        & & Perfect Fifth & & & \\
        & & Major Seventh & & & \\
        & & & & & \\
    \hline
\end{tabular}

\begin{music}
    \startextract
    \Notes \zw c\zw e\zw g\zw i\en
    \endextract
\end{music}

As an example
\raisebox{0ex}[5ex][4ex]%
{\musicintext{\notes\wh{gg}\en}}

\end{document}